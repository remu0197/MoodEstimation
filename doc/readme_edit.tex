\documentclass[10pt]{jsarticle}
\usepackage{cases}
\usepackage{bm}
\usepackage[dvipdfmx]{graphicx}
\usepackage{indentfirst}
\begin{document}
\begin{center}
{\LARGE 発話時間の特徴量選択について}\\
\end{center}
\section{特徴量}
\subsection{状態集合の定義}
\noindent
\begin{tabbing}
\hspace{20mm} \= \hspace{15mm} \kill
\ \ \ \ $S_1$ \>:話者Aが発話している状態の集合\\
\ \ \ \ $S_2$ \>:話者Bが発話している状態の集合\\
\ \ \ \ $S_3$\> :二人の話者が同時に発話している状態の集合\\
\ \ \ \ $S_4$\>:どちらの話者も発話していない状態の集合\\
\end{tabbing}
\subsection{統計量(特徴量番号1 - 24)}
\begin{tabbing}
\hspace{40mm} \= \hspace{15mm} \kill
\ \ \ \ $mean(S)$ \>:Sの平均\\
\ \ \ \ $var(S)$ \>:Sの分散\\
\ \ \ \ $min(S)$\> :Sの最小値\\
\ \ \ \ $max(S)$\>:Sの最大値\\
\ \ \ \ $count(S)$\>:Sの要素数\\
\ \ \ \ $occupy(S)$ \> :Sの対話全体における占有率\\
\ \ \ \ 候補(使わないかも):\\
\ \ \ \ \ \ \ \ $25p\_tile(S)$ \> :Sの25パーセンタイル\\
\ \ \ \ \ \ \ \ $median(S)$ \> :Sの中央値\\
\ \ \ \ \ \ \ \ $75p\_tile(S)$ \> :Sの75パーセンタイル\\
\end{tabbing}
\subsection{状態間の特徴}
stat:各統計量
\begin{itemize}
\item 話者Aと話者Bの発話の比(特徴量番号 25 - 48)\\
$\frac{stat(S_2)}{stat(S_1)}$, $\frac{stat(S_2)+stat(S_3)}{stat(S_1)+stat(S_3)}$, $\frac{stat(S_1)}{stat(S_1)+stat(S_2)}$, $\frac{stat(S_2)}{stat(S_1) + stat(S_2)}$
\item 発話全体における無音状態の割合(特徴量番号 49 - 54)\\
$\frac{stat(S_4)}{stat(S_1) + stat(S_2) + stat(S_3)}$
\item 全状態における各状態の割合(特徴量番号 55 - 78)\\
$\frac{stat(S_1)}{\sum_i{stat(S_i)}}$, $\frac{stat(S_2)}{\sum_i{stat(S_i)}}$, $\frac{stat(S_3)}{\sum_i{stat(S_i)}}$, $\frac{stat(S_4)}{\sum_i{stat(S_i)}}$, 
\item 話者A(話者B)の発話と同時発話の比(特徴量番号 79 - 114)\\
\ \ \ $\frac{stat(S_3)}{stat(S_1)}$, $\frac{stat(S_3)}{stat(S_1) + stat(S_3)}$, $\frac{stat(S_1)}{stat(S_1) + stat(S_3)}$\\
(\ \ $\frac{stat(S_3)}{stat(S_2)}$, $\frac{stat(S_3)}{stat(S_2) + stat(S_3)}$, $\frac{stat(S_2)}{stat(S_2) + stat(S_3)}$\ \ )
\end{itemize}

\end{document}